\documentclass[9pt]{extarticle}
\usepackage{fullpage}
\usepackage{todonotes}
\usepackage{mathpazo}
\usepackage{amsmath}

\newcommand{\note}[1]{\todo[inline,color=gray!20!white]{#1}}
\newcommand{\R}{\mathbb{R}}
\newcommand{\code}[1]{\texttt{#1}}

\title{Tarea 3}
\author{}
\date{16/04/2025}

\begin{document}
%\fontsize{10}{10}
\maketitle
\note{Esta tarea es individual y se debe entregar en el buzón habilitado para eso en Canvas con plazo máximo de entrega el miércoles 07/05/2025 al inicio de la ayudantía (i.e. 2 semanas de plazo). Tareas entregadas en \LaTeX (que sean comprensibles) tendrán una bonificación de 5 décimas. 

A menos que se indique lo contrario, todas las preguntas valen lo mismo. Para el desarrollo de los problemas, escriba el desarrollo en el mayor detalle posible, y defina todos los objetos matemáticos que use.

La tarea tiene 18 puntos, pero el puntaje máximo será de 10 puntos. Si responde más, su nota será dada por las mejores 10 respuestas. En cada pregunta, el puntaje dado será: 1 punto si está bien, 0.5 puntos si tiene un desarrollo concluyente pero tiene errores, y 0.0 puntos si tiene poco o nada de desarrollo.}

\begin{itemize}
    \item Construir número $e$
    \item Demostrar que $sin(1/x)$ no tiene límite en $x\to 0$. 
    \item Demostrar que $sin(1/x)$ tiene límite superior e inferior en $x\to 0$ y calcularlo. 
    \item Demuestre que el espacio de funciones continuas es un espacio vectorial.
    \item Considere el operador integral $I(f) = \int_0^1 f(s)\, ds$. Usando que $\int f \leq \int |f|$, explique cuales son el dominio y codominio de $I$. Muestre que el operador está bien definido para funciones en el espacio $C([0,1], \R)$. Qué pasa con el espacio $C((0,1), \R)$?
\end{itemize} % preguntas

\vspace{1cm}

\end{document}
