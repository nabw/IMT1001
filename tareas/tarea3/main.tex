\documentclass[9pt]{extarticle}
\usepackage{fullpage}
\usepackage{todonotes}
\usepackage{mathpazo}
\usepackage{amsmath}

\newcommand{\note}[1]{\todo[inline,color=gray!20!white]{#1}}
\newcommand{\R}{\mathbb{R}}
\newcommand{\code}[1]{\texttt{#1}}
\newcommand{\pts}[1]{[{\bf #1 puntos}] }

\title{Tarea 3}
\author{}
\date{16/04/2025}

\begin{document}
%\fontsize{10}{10}
\maketitle
\note{Esta tarea es individual y se debe entregar en el buzón habilitado para eso en Canvas con plazo máximo de entrega el miércoles 07/05/2025 al inicio de la ayudantía (i.e. 3 semanas de plazo con receso en medio). Tareas entregadas en \LaTeX (que sean comprensibles) tendrán una bonificación de 5 décimas. 

A menos que se indique lo contrario, todas las preguntas valen lo mismo. Para el desarrollo de los problemas, escriba el desarrollo en el mayor detalle posible, y defina todos los objetos matemáticos que use.

La tarea tiene 14 puntos, pero el puntaje máximo será de 9 puntos. Si responde más, su nota será dada por las mejores 5 respuestas. En cada pregunta, el puntaje dado será: 1 punto si está bien, 0.5 puntos si tiene un desarrollo concluyente pero tiene errores, y 0.0 puntos si tiene poco o nada de desarrollo.}

\begin{itemize}
    \item\pts{2} Use el Teorema de sucesiones monótonas para demostrar que existe el límite de la sucesión $a_n = (1+1/n)^n$. Este límite es el número de Euler (o constante de Napier) $e$. 
    \item\pts{2} Demuestre que la función $f(x) = \sin(1/x)$ no tiene límite en el punto $x_0 = 0$. 

    \item\pts{2} Demuestre que la función $f(x)=\sin(1/x)$ tiene límite superior e inferior en $x\to 0$, y calcularlos. 

    \item\pts{2} Demuestre que el espacio de funciones continuas reales es un espacio vectorial con las operaciones suma y producto estándar de los números reales.

    \item Considere el operador integral $I(f) = \int_0^1 f(s)\, ds$. Usando que $\int f \leq \int |f|$:
            \begin{itemize}
                \item\pts{2} Muestre que el operador está bien definido en $C([0,1], \R)$, i.e. que no existe ninguna $f$ en $C([0,1],\R)$ tal que $I(f)=\infty$. \emph{Hint: Demuestre por contradicción que las funciones continuas en un intervalo cerrado son acotadas y use esa cota.} 
                \item\pts{1} Muestre que el operador no está bien definido en $C((0,1), \R)$.
                \item\pts{2} Caracterice el dominio de $I$, i.e. el espacio de funciones donde la función está bien definida, en términos de los espacios funcionales que vimos en clases. 
                \item\pts{1} Caracterice el codominio de $I$.
            \end{itemize}
\end{itemize} % preguntas

\vspace{1cm}

\end{document}
