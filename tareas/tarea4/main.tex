\documentclass[9pt]{article}
\usepackage{fullpage}
\usepackage{todonotes}
\usepackage{mathpazo}
\usepackage{amsmath}


\newcommand{\note}[1]{\todo[inline,color=gray!20!white]{#1}}
\newcommand{\R}{\mathbb{R}}
\newcommand{\code}[1]{\texttt{#1}}
\newcommand{\pts}[1]{[{\bf #1 puntos}] }
\newtheorem{theorem}{Theorem}

\title{Tarea 4}
\author{}
\date{07/05/2025}

\usepackage[spanish]{babel}
\begin{document}
\maketitle
\note{Esta tarea es individual y se debe entregar en el buzón habilitado para eso en Canvas con plazo máximo de entrega el miércoles 21/05/2025 al inicio de la ayudantía (i.e. 2 semanas de plazo). Tareas entregadas en \LaTeX (que sean comprensibles) tendrán una bonificación de 5 décimas. 

A menos que se indique lo contrario, todas las preguntas valen lo mismo. Para el desarrollo de los problemas, escriba el desarrollo en el mayor detalle posible, y defina todos los objetos matemáticos que use.

La tarea tiene 16 puntos, pero el puntaje máximo será de 12 puntos. Si responde más, su nota será dada por las mejores respuestas. En cada pregunta, el puntaje dado será: 1 punto si está bien, 0.5 puntos si tiene un desarrollo concluyente pero tiene errores, y 0.0 puntos si tiene poco o nada de desarrollo.}

\begin{itemize}
    \item\pts{2} Dado un espacio abstracto $X$, decimos que un operador $P:X\to X$ es una proyección si se cumple que $P^2 = P$. Demuestre que dado un vector unitario $\vec v$ en $\R^n$, la proyección a lo largo de $\vec v$ y su complemento ortogonal, dados por
            $$ P_v(\vec x) = \vec v\vec v^T \vec x, \qquad P_v^\perp(\vec x) = (I - P_v)\vec x$$
            son efectivamente proyecciones en el sentido abstracto. 

    \item El objetivo de esta pregunta es demostrar que el problema de Cauchy asociado a una EDO tiene garantías de que exista una solución bajo ciertas hipótesis. Para eso, consideramos una función $f:\R_+ \times \R \to \R$ continua y el rectángulo
        $$ R = \{(t,x): t \in [0,T], x \in [u_0-L, u_0+L] \}. $$
        El Teorema que queremos demostrar es el siguiente: 
        \begin{theorem}
            Si $f$ es de clase $C^1$ (continua y con derivada continua) en $R$, luego existe una única función $u=u(t)$ tal que 
            \begin{equation}\label{eq:edo}
                u'(t) = f(t, u) 
            \end{equation}
            que cumple la condición inicial $u(0)=u_0$, y que está definida en un intervalo $[0,t^*]\subset [0,T]$. 
        \end{theorem}
        Descompondremos su demostración en varios pasos. 
        \begin{enumerate}
            \item\pts{1} Demuestre con el Teorema Fundamental del Cálculo que $u$ es solución de \eqref{eq:edo} si y solo si cumple que
                $$ u(t) = u_0 + \int_0^t f(s, u(s))\,ds. $$
            \item\pts{1} Demuestre que como $f$ es $C^1$, se tiene que es Lipschitz en su segundo argumento, i.e. que para todo $t$ existe una constante positiva $C$ tal que 
                    $$ |f(t, x_1) - f(t, x_2)|\leq C |x_1 - x_2|. $$
            \item\pts{3} Considere el operador $T:C(\R,\R)\to C(\R,\R)$ definido como
                        $$ T(y)(t) = u_0 + \int_0^t f(s, y(s))\,s. $$
                    Muestre que este operador es una contracción para $t$ suficientemente cercano a $0$, i.e. que necesita que $t\leq t^*$ para algún $t^*$. 
            \item\pts{2} A partir de los resultados anteriores, use el Teorema de Punto Fijo de Banach para mostrar que existe un punto fijo de $T$ que resuelve \eqref{eq:edo}. 
        \end{enumerate}
    \item Considere la función discontinua $f:[-1,1]\to \R$ dada por
            $$ f(x) = I_{(-1/2,1/2)}(x). $$
            \begin{itemize}
                \item\pts{2} Encuentre los coeficientes de Fourier $a_0, \alpha_n, \beta_n$ tales que 
                    $$ f(x) = a_0 + \sum_{n=1}^\infty \alpha_n \sin(n\pi x) + \beta_n \cos(n\pi x). $$
                    Para hacerlo, demuestre que las funciones $1,\sin(k\pi x), \cos(k\pi x)$ son ortogonales en $L^2$ y luego calcule el producto entre dichas funciones y $f$.
                \item\pts{3} Para los coeficientes calculados, considere la serie truncada
                    $$ f_N(x) = a_0 + \sum_{n=1}^N \alpha_n \sin(n\pi x) + \beta_n \cos(n\pi x). $$
                    Programe esta serie y grafíquela para los valores $N\in \{1,5,10,50,100\}$.
                \item\pts{2} Estudie el valor de la serie aproximante en los puntos $\{-1/2,1/2\}$ y comente el resultado en términos de la continuidad de las funciones $f$ y $f_N$. Elija un $N$ grande para que ambas funciones sean parecidas. 
            \end{itemize}
\end{itemize} % preguntas

\vspace{1cm}

\end{document}
