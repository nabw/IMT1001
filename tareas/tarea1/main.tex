\documentclass{article}
\usepackage{fullpage}
\usepackage{todonotes}
\usepackage{mathpazo}

\newcommand{\note}[1]{\todo[inline,color=gray!20!white]{#1}}
\newcommand{\R}{\mathbb{R}}
\newcommand{\code}[1]{\texttt{#1}}

\title{Tarea 1}
\author{}

\begin{document}
\maketitle
\note{Esta tarea es individual y se debe entregar en el buzón habilitado para eso en Canvas con plazo máximo de entrega el miércoles 02/04/2025 al inicio de la ayudantía (i.e. 2 semanas de plaz). Tareas entregadas en \LaTeX (que sean comprensibles) tendrán una bonificación de 5 décimas. 

A menos que se indique lo contrario, todas las preguntas valen lo mismo. Para el desarrollo de los problemas, escriba el desarrollo en el mayor detalle posible, y defina todos los objetos matemáticos que use.

En la tarea se presentan diez problemas, pero el puntaje máximo será de 7 puntos. Si responde más, su nota será dada por las mejores 7 respuestas. En cada pregunta, el puntaje dado será: 1 punto si está bien, 0.5 puntos si tiene un desarrollo concluyente pero tiene errores, y 0.0 puntos si tiene poco o nada de desarrollo.}

\begin{itemize}
    \item[\bf (1pt)]  Muestre que, dadas dos proposiciónes $P, Q$, se tiene que $P\Rightarrow Q$ y $\neg P \vee Q$ son equivalentes. 
    \item Para un número real $x$ y dos enteros $m,n$, demuestre por inducción (a partir de la definición por inducción) las leyes de exponentes: 
        \begin{itemize}
            \item[\bf (1pt)] $x^ny^n = (xy)^n$
            \item[\bf (1pt)] $x^{m+n} = x^mx^n$
            \item[\bf (1pt)] $(x^m)^n = x^{mn}$
        \end{itemize}
    \item Los números reales tienen una operación suma $+:\R\times \R\to \R$ y un producto $\cdot: \R\times\R\to \R$, que usualmente se escriben como $a+b$ y $ab$ (donde se omite el punto)  con las siguientes propiedades:
        \begin{itemize}
            \item (Conmutatividad) $a+b=b+a$ y $ab=ba$
            \item (Asociatividad) $(a+b)+c = a+(b+c)$ y $(ab)c = a(bc)$
            \item (Distributividad) $a(b+c) = ab+ac$ y $(a+b)c = ac+bc$
            \item (Neutro aditivo) Existe un elemento '0' tal que $a+0 = 0+a = a$
            \item (Neutro multiplicativo) Existe un elemento '1' tal que $a1 = 1a = a$
            \item (Resta) La ecuación $a+x =0$ tiene una única solución $x=-a$, por lo que $a+x = b$ tiene como solución $x = b + (-a) = b-a$.
            \item (División) Si $a\neq 0$, la ecuación $ax=b$ tiene una única solución $x = b/a = ba^{-1}$. 
        \end{itemize}

        Usando estas propiedades, demuestre las siguientes afirmaciones: 
        \begin{itemize}
            \item[\bf (1pt)]  $a \cdot 0 = 0 \cdot a = 0$
            \item[\bf (1pt)] $(-a)b = -ab = a(-b)$
            \item[\bf (1pt)] $(-a)(-b) = ab$
        \end{itemize}

    \item[\bf (1pt)] Muestre por contradicción que no existe un número entero que sea más grande que todos los demás.
    \item[\bf (1pt)] Demuestre la desigualdad de Bernoulli: Para todo entero no negativo y $x > -1$, se tiene que
            $$ (1+x)^n \geq 1 + nx. $$
        \item[\bf (1pt)] Elija un problema de la sección de problemas de la parte I del libro de Eccles (página 53) que no sea uno de los de esta tarea y que le parezca interesante. Explique por qué le parece interesante y resuélvalo. 

\end{itemize} % preguntas

\vspace{1cm}

\end{document}
