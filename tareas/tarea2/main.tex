\documentclass[9pt]{extarticle}
\usepackage{fullpage}
\usepackage{todonotes}
\usepackage{mathpazo}
\usepackage{amsmath}

\newcommand{\note}[1]{\todo[inline,color=gray!20!white]{#1}}
\newcommand{\R}{\mathbb{R}}
\newcommand{\code}[1]{\texttt{#1}}

\title{Tarea 2}
\author{}
\date{02/04/2025}

\begin{document}
%\fontsize{10}{10}
\maketitle
\note{Esta tarea es individual y se debe entregar en el buzón habilitado para eso en Canvas con plazo máximo de entrega el miércoles 16/04/2025 al inicio de la ayudantía (i.e. 2 semanas de plazo). Tareas entregadas en \LaTeX (que sean comprensibles) tendrán una bonificación de 5 décimas. 

A menos que se indique lo contrario, todas las preguntas valen lo mismo. Para el desarrollo de los problemas, escriba el desarrollo en el mayor detalle posible, y defina todos los objetos matemáticos que use.

La tarea tiene 18 puntos, pero el puntaje máximo será de 10 puntos. Si responde más, su nota será dada por las mejores 10 respuestas. En cada pregunta, el puntaje dado será: 1 punto si está bien, 0.5 puntos si tiene un desarrollo concluyente pero tiene errores, y 0.0 puntos si tiene poco o nada de desarrollo.}

\begin{itemize}
    \item[\bf{[1 pt c/u]}] Muestre las siguientes identidades:
            $$ \{x\in \R: x^2 - 3x + 2 < 0 \} = \{x\in R: x< 2\} \cap \{x\in \R: x > 1\} $$
            $$\{x\in \R: x^2 - 3x+2 > 0 \} = \{x\in \R: x>2\} \cup \{x\in R: x<1\} $$

    \item [\bf{[1 pt c/u]}] Demuestre las leyes de De Morgan.

    \item[\bf{[1 pt c/u]}] Demuestre o de un contraejemplo de las siguientes afirmaciones:
        \begin{itemize}
            \item $\forall x\in \R, \exists y \in \R, x+y > 0 $
            \item $\forall x\in \R, \exists y \in \R, (x+y > 0 \text{ y } x + y = 0 )$
            \item $\forall x\in \R, \exists y \in \R, x+y > 0$ y $\forall x \in \R, \exists y \in \R,  x + y = 0$

        \end{itemize}

    \item[\bf{[2 pt]}] Considere las funciones $f,g:\R\to\R$ dadas por $f(x) = x^2$ y $g(x) = x^2-1$. Caracterice el conjunto 
        $$ \{x \in \R: f\circ g(x) = g\circ f(x) \}. $$

    \item[\bf{[1 pt c/u]}] Definimos la función \emph{característica} o \emph{indicatriz} de un conjunto $A\in \mathcal P(U)$ como $I_A:X\to \{0,1\}$ dada por
        $$ I_A(x) = \begin{cases} 0 & x\not\in A \\ 1 & x \in A\end{cases}. $$
        \begin{itemize}
            \item Demuestre que $I_A(x)I_B(x)=I_{A\cap B}(x)$.
            \item Encuentre la función indicatriz de $A\cup B$ en términos de $I_A$ y $I_B$. \emph{Hint: use la pregunta anterior}.
        \end{itemize}

    \item[\bf{[2 pt]}]  Considere dos funciones sobreyectivas $f:X\to Y$ y $g:Y\to Z$. Demuestre que $g\circ f$ es sobreyectiva. 


    \item[\bf{[2 pt]}]  Muestre que la siguiente función es una biyección y escriba su inversa: 
                $$ f(x) = \begin{cases} x^2 & x\geq 0 \\ -x^2 & x < 0 \end{cases}. $$

    \item[\bf{[3 pt]}] Demuestre que una función es biyectiva si y solo si es invertible.
\end{itemize} % preguntas

\vspace{1cm}

\end{document}
