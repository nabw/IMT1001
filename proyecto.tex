\documentclass{article}
\usepackage{fullpage}
\usepackage{amsmath,amsfonts}

\title{Proyecto: Solución numérica de ecuaciones diferenciales}
\author{Nicolás Barnafi (\texttt{nicolas.barnafi@uc.cl})}
\date{}
\begin{document}
\maketitle


\noindent Considere la ecuación de onda, dada por la siguiente ecuación espacio-temporal: 
        $$ \frac{d^2 u}{dt^2} - \mu \frac{d^2 u}{dx^2} = 0,$$
para los puntos $x$ en $(0,1)$, instantes $t$ en $(0,T)$, dada la condición de borde $u(0, t)=u(1, t)=0$ para cada instante $t$, y con condiciones iniciales $u(x, 0) = u_0(x)$ y $\frac{d u}{dt}(x, 0) = v_0(x)$. 

    \begin{enumerate}
        \item (1.0) Discretice el sistema en espacio de manera uniforme en $N$ subintervalos, calcule el ancho $h$ de los intervalos en función de $N$ y muestre el sistema de ecuaciones diferenciales resultante. Muestre detalladamente los objetos matemáticos involucrados (matrices, vectores y sus dimensiones), y cómo discretizaría las condiciones iniciales. 
        \item (1.0) Discretice en el tiempo la ecuación diferencial del punto anterior para un paso temporal $\Delta t$. Muestre los sistemas que obtendría con una aproximación (i) implícita (ii) explícita y (iii) trapezoidal.
        \item (1.5) Encuentre una configuración estable de discretización en tiempo, $h$ y $\Delta t$ tal que le permita simular el problema para $\mu=0.01$ hasta un horizonte temporal de $T=8$, usando como condiciones iniciales las funciones $v_0(x) = 0$ y 
            $$ u_0(x) = \begin{cases} 1 & x\in [0.4, 0.6] \\   0 & e.o.c \end{cases}. $$
        *e.o.c: en otro caso. El resultado obtenido debiese validar el nombre de la ecuación. Muestre la solución en algunos instantes que permitan apreciar su evolución. 
        \item (1.5) Considere el problema del punto anterior con una discretización implícita y con una trapezoidal, que implican la solución de un sistema lineal. Muestre la evolución de iteraciones de Gauss-Seidel que require en cada paso temporal (para cada una de las dos discretizaciones en tiempo pedidas) para resolver el problema para una tolerancia relativa de $10^{-6}$. Qué problema es más difícil de resolver? 
        \item (1.0) El esquema implícito es más difícil de resolver en cada paso temporal, pero también es más estable que el explícito. Intente considerar un esquema explícito con paso $\Delta t_1$ y uno implícito con paso $\Delta t_2$ tales que las soluciones sean parecidas (donde inevitablmente el paso del esquema explícito tendrá que ser más pequeño para obtener una solución estable). Considerando el horizonte temporal $T=8$, mida el tiempo que le tome resolver cada problema y discuta qué método es efectivamente más rápido para resolver la ecuación de onda. 
    \end{enumerate}

\vspace{1cm}
*Para el desarrollo de la tarea, se pueden usar todos los códigos que vimos durante el curso. Si le falta información en el enunciado para resolver un problema, asuma justificadamente los datos que faltan ($h$, $\Delta$).
\end{document}
