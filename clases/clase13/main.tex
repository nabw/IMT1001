\documentclass[12pt,aspectratio=169,xcolor=dvipsnames]{beamer}
\usetheme{SimplePlus}
\usepackage{booktabs}
\usepackage{tikz}
\usepackage{pgfplots}
\usepackage{mathtools}

\newcommand{\R}{\mathbb{R}}
\newcommand{\N}{\mathbb{N}}

\title[short title]{Clase 13 Espacios normados, límites y puntos fijos}
\subtitle{}
\author[NA Barnafi] {Nicolás Alejandro Barnafi Wittwer}
\institute[UC|CMM] 
{
    Pontificia Universidad Católica de Chile \\
    Centro de Modelamiento Matemático
}

\titlegraphic{
    \vspace{-1.8cm}
    \begin{flushright}
      \includegraphics[height=2.5cm]{../images/puc.png} 
    \end{flushright}
}

\date{21/04/2025}
%\setbeamercovered{transparent}

\begin{document}
%%%%%%%%%%%%%%%%%%%%%%%%%%%%%%%%%%%%%%%%%%%%%%%%%%%%%%%
\begin{frame}
    \maketitle
\end{frame}
%%%%%%%%%%%%%%%%%%%%%%%%%%%%%%%%%%%%%%%%%%%%%%%%%%%%%%%
\begin{frame}{Clase de hoy}
    \begin{itemize}
        \item Espacios vectoriales normados
        \item Ejemplos
        \item Límites abstractos
        \item Clase Lipschitz y operadores de punto fijo
    \end{itemize}

    \vspace{1cm}
    \newref{Abbott, Understanding Analysis.}
    \newref{Tao, Analysis I \& II}
\end{frame}
%%%%%%%%%%%%%%%%%%%%%%%%%%%%%%%%%%%%%%%%%%%%%%%%%%%%%%%
\begin{frame}\frametitle{Espacio vectorial normado}
    Un espacio vectorial normado $V$ es un espacio vectorial que tiene además una función \emph{norma} $\|\cdot\|:V \to \R$ tal que: 
    \begin{itemize}
        \item Es no-negativa $\|x\|\geq 0\quad\forall x \in V$
        \item Es definida positiva: $\|x\|=0$ ssi $x=0$ 
        \item Es absolutamente homogénea: $\|\lambda x\| = |\lambda| \|x\|$, para $\lambda \in \R, x\in V$
        \item Desigualdad triangular: $\|x+y\| \leq \|x\| + \|y \|$, $\forall x,y \in V$. 
    \end{itemize}

\end{frame}
%%%%%%%%%%%%%%%%%%%%%%%%%%%%%%%%%%%%%%%%%%%%%%%%%%%%%%%%%%%%%%%
\begin{frame}{Ejemplos}
    \begin{itemize}
        \item $\R$ con la norma dada por el valor absoluto.
        \item $\R^n$ con la norma dada por la norma de vector.
        \item Espacio de funciones continuas $C(\R,\R)$ con norma del supremo: 
            $$ \| f \|_{C(\R,\R)} = \sup_{x\in \R}|f(x)|. $$
    \end{itemize}
    \idea{Demostrar}
\end{frame}
%%%%%%%%%%%%%%%%%%%%%%%%%%%%%%%%%%%%%%%%%%%%%%%%%%%%%%%%%%%%%%%
\begin{frame}{Límites abstractos}
    \begin{block}{}
        Sean $X,Y$ dos espacios vectoriales normados y un operador $T:X\to Y$. Decimos que el límite de $T$ cuando $x\in X$ va a $\bar x\in X$ es $L\in Y$ si :
        $$ \forall \epsilon >0, \exists \delta>0: \|x- \bar x\|_X< \delta \Rightarrow \|T(x) - L\|_Y$$
    \end{block}
    \idea{Si $f$ en $C(\R,\R)$ tiende a $g$ en $C(\R,\R)$, luego $T(f)=f^2$ tiende a $g^2$}
\end{frame}
%%%%%%%%%%%%%%%%%%%%%%%%%%%%%%%%%%%%%%%%%%%%%%%%%%%%%%%
\begin{frame}{Continuidad Lipschitz}
    \begin{block}{Continuidad Lipschitz}
        $T:X\to Y$ es Lipschitz si existe $L>0$ tal que
        $$ \| T(x) - T(y) \|_Y \leq L\|x-y\|_X \quad \forall x,y \in X$$
    \end{block}
    Si $L<1$, decimos que $T$ es una \emph{contracción}. 
\end{frame}
%%%%%%%%%%%%%%%%%%%%%%%%%%%%%%%%%%%%%%%%%%%%%%%%%%%%%%%
\begin{frame}{Ejemplo: Funciones con derivada acotada}
    Todas las funciones con derivada acotada $|f'(x)|<M$ son Lipschitz.
    Usaremos Teorema del Valor Medio: Si $f$ es $C^1(\R,\R)$; 
            $$ \exists \xi \in (a,b): \quad f(b) - f(a) = f'(\xi)(b-a) $$
    \pause \begin{proof}
        De la desigualdad de TVM, para todos $x,y$ en $\R$:
        \begin{align*}
            |f(x) - f(y)| &= |f'(\xi)(b-a)|, \quad \xi \in (x,y) \\
                          &\leq |f'(\xi)| |x-y|, \quad \text{derivada acotada} \\
                        &\leq M |x-y|.
        \end{align*}
    \end{proof}
\end{frame}
%%%%%%%%%%%%%%%%%%%%%%%%%%%%%%%%%%%%%%%%%%%%%%%%%%%%%%%
\begin{frame}{Puntos fijos}
    Dada una función $f:X\to X$, decimos que $x$ es punto fijo de $f$ si 
        $$ x = f(x) $$
\end{frame}
%%%%%%%%%%%%%%%%%%%%%%%%%%%%%%%%%%%%%%%%%%%%%%%%%%%%%%%%
\begin{frame}{Ejemplos}
    Veamos si las siguientes funciones inducen puntos fijos:
    \begin{itemize}
        \item $f(x)=x^p$
        \item $f(x)=\sin x$
        \item $f(x) = e^{ax}$
    \end{itemize}
\idea{Código}

\idea{Tarea: Demostrar si son contracciones.}
\end{frame}
%%%%%%%%%%%%%%%%%%%%%%%%%%%%%%%%%%%%%%%%%%%%%%%%%%%%%%
\begin{frame}{Ejemplo notable: Función logística}
    $$ f(x) = ax (1-x) $$
    \idea{Código}
\end{frame}
%%%%%%%%%%%%%%%%%%%%%%%%%%%%%%%%%%%%%%%%%%%%%%%%%%%%%%
\begin{frame}{Teorema de punto fijo de Banach}
    \begin{block}{}
        Sea $T:X\to X$ una contracción continua. Luego, existe un único punto fijo de $T$. 
    \end{block}
    \idea{Tarea}
\end{frame}
%%%%%%%%%%%%%%%%%%%%%%%%%%%%%%%%%%%%%%%%%%%%%%%%%%%%%%%
\begin{frame}{Aplicación importantísima}
    Si el lado derecho de la siguiente ecuación diferencial es suficientemente suave: 
        $$ \frac{d f }{dt} = F(t, f), $$
    entonces existe una única $f$ que satisface la ecuación dada una condición inicial. 
\end{frame}
%%%%%%%%%%%%%%%%%%%%%%%%%%%%%%%%%%%%%%%%%%%%%%%%%%%%%%%
\begin{frame}
    \maketitle
\end{frame}
%%%%%%%%%%%%%%%%%%%%%%%%%%%%%%%%%%%%%%%%%%%%%%%%%%%%%%%
\end{document}
