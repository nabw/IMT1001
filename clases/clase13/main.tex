\documentclass[12pt,aspectratio=169,xcolor=dvipsnames]{beamer}
\usetheme{SimplePlus}
\usepackage{booktabs}
\usepackage{tikz}
\usepackage{pgfplots}
\usepackage{mathtools}

\newcommand{\R}{\mathbb{R}}
\newcommand{\N}{\mathbb{N}}

\title[short title]{Clase 13 Normas, límites y puntos fijos}
\subtitle{}
\author[NA Barnafi] {Nicolás Alejandro Barnafi Wittwer}
\institute[UC|CMM] 
{
    Pontificia Universidad Católica de Chile \\
    Centro de Modelamiento Matemático
}

\titlegraphic{
    \vspace{-1.8cm}
    \begin{flushright}
      \includegraphics[height=2.5cm]{../images/puc.png} 
    \end{flushright}
}

\date{16/03/2024}
%\setbeamercovered{transparent}

\begin{document}
%%%%%%%%%%%%%%%%%%%%%%%%%%%%%%%%%%%%%%%%%%%%%%%%%%%%%%%
\begin{frame}
    \maketitle
\end{frame}
%%%%%%%%%%%%%%%%%%%%%%%%%%%%%%%%%%%%%%%%%%%%%%%%%%%%%%%
\begin{frame}{Clase de hoy}
    \begin{itemize}
        \item Espacios vectoriales normados
        \item Ejemplos
        \item Límites abstractos
        \item Clase Lipschitz y operadores de punto fijo
    \end{itemize}

    \vspace{1cm}
    \newref{Abbott, Understanding Analysis.}
    \newref{Tao, Analysis I \& II}
\end{frame}
%%%%%%%%%%%%%%%%%%%%%%%%%%%%%%%%%%%%%%%%%%%%%%%%%%%%%%%
\begin{frame}\frametitle{Espacio vectorial normado}
    Un espacio vectorial normado $V$ es un espacio vectorial que tiene además una función \emph{norma} $\|\cdot\|:V \to \R$ tal que: 
    \begin{itemize}
        \item Es no-negativa $\|x\|\geq 0\quad\forall x \in V$
        \item Es definida positiva: $\|x\|=0$ ssi $x=0$ 
        \item Es absolutamente homogénea: $\|\lambda x\| = |\lambda| \|x\|$, para $\lambda \in \R, x\in V$
        \item Desigualdad triangular: $\|x+y\| \leq \|x\| + \|y \|$, $\forall x,y \in V$. 
    \end{itemize}

\end{frame}
%%%%%%%%%%%%%%%%%%%%%%%%%%%%%%%%%%%%%%%%%%%%%%%%%%%%%%%%%%%%%%%
\begin{frame}{Ejemplos}
    \begin{itemize}
        \item $\R$ con la norma dada por el valor absoluto.
        \item $\R^n$ con la norma dada por la norma de vector.
        \item Espacio de funciones continuas $C(\R,\R)$ con norma del supremo: 
            $$ \| f \|_{C(\R,\R)} = \sup_{x\in \R}|f(x)|. $$
    \end{itemize}
    \idea{Demostrar}
\end{frame}
%%%%%%%%%%%%%%%%%%%%%%%%%%%%%%%%%%%%%%%%%%%%%%%%%%%%%%%%%%%%%%%
\begin{frame}{Límites}
    \begin{block}{}
        Sean $X,Y$ dos espacios vectoriales normados y un operador $T:X\to Y$. Decimos que el límite de $T$ cuando $x\in X$ va a $\bar x\in X$ es $L\in Y$ si :
        $$ \forall \epsilon >0, \exists \delta>0: \|x- \bar x\|_X< \delta \Rightarrow \|T(x) - L\|_Y$$
    \end{block}
\end{frame}
%%%%%%%%%%%%%%%%%%%%%%%%%%%%%%%%%%%%%%%%%%%%%%%%%%%%%%%
\begin{frame}{Continuidad Lipschitz}
\end{frame}
%%%%%%%%%%%%%%%%%%%%%%%%%%%%%%%%%%%%%%%%%%%%%%%%%%%%%%%
\begin{frame}
    Ejemplos
\end{frame}
%%%%%%%%%%%%%%%%%%%%%%%%%%%%%%%%%%%%%%%%%%%%%%%%%%%%%%%
\begin{frame}{Punto fijo}
\end{frame}
%%%%%%%%%%%%%%%%%%%%%%%%%%%%%%%%%%%%%%%%%%%%%%%%%%%%%%%
\begin{frame}
    \maketitle
\end{frame}
%%%%%%%%%%%%%%%%%%%%%%%%%%%%%%%%%%%%%%%%%%%%%%%%%%%%%%%
\end{document}
