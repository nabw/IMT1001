\documentclass[12pt,aspectratio=169,xcolor=dvipsnames]{beamer}
\usetheme{SimplePlus}
\usepackage{booktabs}
\usepackage{tikz}
\usepackage{pgfplots}
\usepackage{mathtools}

\newcommand{\R}{\mathbb{R}}
\newcommand{\N}{\mathbb{N}}

\title[short title]{Clase 14 Operadores de punto fijo}
\subtitle{}
\author[NA Barnafi] {Nicolás Alejandro Barnafi Wittwer}
\institute[UC|CMM] 
{
    Pontificia Universidad Católica de Chile \\
    Centro de Modelamiento Matemático
}

\titlegraphic{
    \vspace{-1.8cm}
    \begin{flushright}
      \includegraphics[height=2.5cm]{../images/puc.png} 
    \end{flushright}
}

\date{23/04/2025}
%\setbeamercovered{transparent}

\begin{document}
%%%%%%%%%%%%%%%%%%%%%%%%%%%%%%%%%%%%%%%%%%%%%%%%%%%%%%%
\begin{frame}
    \maketitle
\end{frame}
%%%%%%%%%%%%%%%%%%%%%%%%%%%%%%%%%%%%%%%%%%%%%%%%%%%%%%%
\begin{frame}{Clase de hoy}
    \begin{itemize}
        \item Definir operadores de punto fijo
        \item Ver muchos ejemplos
    \end{itemize}
\end{frame}
%%%%%%%%%%%%%%%%%%%%%%%%%%%%%%%%%%%%%%%%%%%%%%%%%%%%%%%
\begin{frame}{Problemas de punto fijo}
    \begin{block}{Problema de punto fijo}
        Dado $T:X\to X$, hallar $x$ en $X$ tal que
            $$ x = T(x).$$
        También se puede ver como verificar que es cierto que $\exists x\in X: x=T(x)$.
    \end{block}

    \pause \begin{block}{Problema de búsqueda de raíces}
        Dado $T:X\to X$, hallar $x$ en $X$ tal que
            $$ T(x)=0.$$
    \end{block}
    \idea{Demostrar que son equivalentes.}
\end{frame}
%%%%%%%%%%%%%%%%%%%%%%%%%%%%%%%%%%%%%%%%%%%%%%%%%%%%%%%
\begin{frame}{Ejemplos}
    \begin{itemize}
        \item<+-> Hallar punto fijo de $f(x) = x^2$
        \item<+-> Hallar punto fijo de $f(x) = e^{-x}$
        \item<+-> Hallar puntos fijos de operador derivada $D f= f'$
        \item<+-> Hallar punto fijo de $f(x) = \sin x \cos (75 x) + e^{\tan x^2}$
    \end{itemize}
    \pause \idea{Idea: cuando el papel no funciona, probamos el computador.}
\end{frame}
%%%%%%%%%%%%%%%%%%%%%%%%%%%%%%%%%%%%%%%%%%%%%%%%%%%%%%%
\begin{frame}{Iteraciones de punto fijo}
    \begin{block}{}
        Dado un operador $T:X\to X$ y un punto inicial $x_0$, definimos una iteración de punto fijo como la sucesión dada por
            $$ x^{k+1} = T(x^k)$$
    \end{block}
    Si la sucesión $(x^k)_k = (T(x^{k-1}))_k$ converge, entonces converge a un punto fijo de $T$.

    \pause \alertGreen{Para verificar esto, usamos el Teorema de punto fijo de Banach.}
\end{frame}
%%%%%%%%%%%%%%%%%%%%%%%%%%%%%%%%%%%%%%%%%%%%%%%%%%%%%%%
\begin{frame}{Ejemplo}
    $$f(x) = \omega \sin x $$
    Por Teorema de valor medio, sabemos que $f(x) - f(y)=f'(\chi)(x-y)$, luego... 
    \pause 
    \begin{equation*}
        \begin{aligned}
            |f(x) - f(y)| &= |f'(\chi)||x-y| && \textcolor{gray}{[\text{usamos } \sin x \leq 1]}\\
                        &\leq \omega |x-y|. 
        \end{aligned}
    \end{equation*}
    Por lo tanto, es una contracción si $\omega<1$. Por TPF, entonces tiene \emph{un único} punto fijo.  \pause \alert{Y si $\omega\geq1$? }

    \pause \idea{Código}
\end{frame}
%%%%%%%%%%%%%%%%%%%%%%%%%%%%%%%%%%%%%%%%%%%%%%%%%%%%%%%
\begin{frame}
    \maketitle
\end{frame}
%%%%%%%%%%%%%%%%%%%%%%%%%%%%%%%%%%%%%%%%%%%%%%%%%%%%%%%
\end{document}
