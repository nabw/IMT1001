\documentclass[14pt,aspectratio=169,xcolor=dvipsnames]{beamer}
\usetheme{SimplePlus}
\usepackage{booktabs}

\newcommand{\R}{\mathbb{R}}
\newcommand{\N}{\mathbb{N}}

\title[short title]{Clase 11 Procesos límite: funciones}
\subtitle{}
\author[NA Barnafi] {Nicolás Alejandro Barnafi Wittwer}
\institute[UC|CMM] 
{
    Pontificia Universidad Católica de Chile \\
    Centro de Modelamiento Matemático
}

\titlegraphic{
    \vspace{-1.8cm}
    \begin{flushright}
      \includegraphics[height=2.5cm]{../images/puc.png} 
    \end{flushright}
}

\date{14/04/2025}
%\setbeamercovered{transparent}

\begin{document}
%%%%%%%%%%%%%%%%%%%%%%%%%%%%%%%%%%%%%%%%%%%%%%%%%%%%%%%
\begin{frame}
    \maketitle
\end{frame}
%%%%%%%%%%%%%%%%%%%%%%%%%%%%%%%%%%%%%%%%%%%%%%%%%%%%%%%
\begin{frame}{Clase de hoy}
    \begin{itemize}
        \item Sucesiones
        \item Límite superior/inferior
    \end{itemize}

    \vspace{1cm}
    \newref{Spivak, Cálculo.}
    \newref{https://ocw.mit.edu/courses/18-100a-real-analysis-fall-2020/mit18\_100af20\_lec92.pdf}
\end{frame}
%%%%%%%%%%%%%%%%%%%%%%%%%%%%%%%%%%%%%%%%%%%%%%%%%%%%%%%
\begin{frame}{Límites de funciones}
    \begin{small}
        Decimos que una función converge a $L$ cuando $x$ va a $x_0$ cuando...\footnote{Ambas son equivalentes}
        \begin{block}{Epsilon-delta ($\epsilon-\delta$)}
            $$\forall \epsilon>0, \exists \delta >0, \forall x\in \text{Dom}(f), |x-x_0|<\delta \Rightarrow |f(x) - f(x_0)| < \epsilon $$
        \end{block}

        \begin{block}{Sucesiones}
            $$\forall (x_n)_n \subset \text{Dom}(f)\setminus \{x_0\}, \lim_n x_n = x_0 \Rightarrow \lim_n f(x_m) = L $$
        \end{block}
        \idea{Demostrar con ambas definiciones que $\lim_{x\to 2}x^2=4$}
    \end{small}
\end{frame}
%%%%%%%%%%%%%%%%%%%%%%%%%%%%%%%%%%%%%%%%%%%%%%%%%%%%%%%
\darkSlide{En general, las demostraciones con sucesiones son más fáciles.}
%%%%%%%%%%%%%%%%%%%%%%%%%%%%%%%%%%%%%%%%%%%%%%%%%%%%%%%
\begin{frame}{Límite superior / inferior}
    Dada sucesión $(x_n)_n$:
    \begin{block}{Límite superior}
        $$\limsup_{n\to\infty}x_n = \lim_{n\to\infty}\left(\sup_{k\geq n}x_n\right) $$
    \end{block}

    \begin{block}{Límite inferior}
        $$\limsup_{n\to\infty}x_n = \lim_{n\to\infty}\left(\inf_{k\geq n}x_n\right)$$
    \end{block}
    \idea{Encuentre el liminf y el limsup de $a_n=(-1)^n$.}
\end{frame}
%%%%%%%%%%%%%%%%%%%%%%%%%%%%%%%%%%%%%%%%%%%%%%%%%%%%%%%
\begin{frame}[t]{Ejercicio}
    \begin{block}{Límite superior}
        $$\limsup_{n\to\infty}x_n = \lim_{n\to\infty}\left(\sup_{k\geq n}x_n\right) $$
    \end{block}
    \begin{block}{Teorema de sucesiones monótonas}
        Se $(a_n)_n$ es una sucesión monótona creciente y acotada superiormente, entonces es convergente y $ \lim_{n \to \infty}a_n = \sup_n a_n $
    \end{block}
    Demuestre usando el teorema de sucesiones monótonas que el limsup siempre existe. Extienda su conclusión para el liminf. 
\end{frame}
%%%%%%%%%%%%%%%%%%%%%%%%%%%%%%%%%%%%%%%%%%%%%%%%%%%%%%%
\begin{frame}{Propiedades}
    \begin{enumerate}
        \item $\liminf_n a_n \leq \limsup_n a_n$
        \item Relación entre límites: 
            $$ \lim_n a_n = L \Leftrightarrow \liminf_n a_n = L = \limsup_n a_n $$
    \end{enumerate}
    \idea{Demostrar la primera}
\end{frame}
%%%%%%%%%%%%%%%%%%%%%%%%%%%%%%%%%%%%%%%%%%%%%%%%%%%%%%%
\begin{frame}
    \maketitle
\end{frame}
%%%%%%%%%%%%%%%%%%%%%%%%%%%%%%%%%%%%%%%%%%%%%%%%%%%%%%%
\end{document}
