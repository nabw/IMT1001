\documentclass[14pt,aspectratio=169,xcolor=dvipsnames]{beamer}
\usetheme{SimplePlus}
\usepackage{booktabs}

\newcommand{\R}{\mathbb{R}}
\newcommand{\N}{\mathbb{N}}

\title[short title]{Clase 10 Procesos límite}
\subtitle{}
\author[NA Barnafi] {Nicolás Alejandro Barnafi Wittwer}
\institute[UC|CMM] 
{
    Pontificia Universidad Católica de Chile \\
    Centro de Modelamiento Matemático
}

\titlegraphic{
    \vspace{-1.8cm}
    \begin{flushright}
      \includegraphics[height=2.5cm]{../images/puc.png} 
    \end{flushright}
}

\date{14/04/2025}
%\setbeamercovered{transparent}

\begin{document}
%%%%%%%%%%%%%%%%%%%%%%%%%%%%%%%%%%%%%%%%%%%%%%%%%%%%%%%
\begin{frame}
    \maketitle
\end{frame}
%%%%%%%%%%%%%%%%%%%%%%%%%%%%%%%%%%%%%%%%%%%%%%%%%%%%%%%
\begin{frame}{Clase de hoy}
    \begin{itemize}
        \item Sucesiones
        \item Límites de sucesiones
        \item Límites de funciones
        \item Límite superior/inferior
    \end{itemize}

    \vspace{1cm}
    \newref{Spivak, Cálculo.}
    \newref{https://ocw.mit.edu/courses/18-100a-real-analysis-fall-2020/mit18\_100af20\_lec92.pdf}
\end{frame}
%%%%%%%%%%%%%%%%%%%%%%%%%%%%%%%%%%%%%%%%%%%%%%%%%%%%%%%
\begin{frame}{Sucesiones}
    Una sucesión es una función $a:\mathbb N\to \R$, que podemos escribir $a(n)$ pero comúnmente denotamos $a_n$. Se definen casi siempre por recursión o por el símbolo de evaluación:
    \begin{itemize}
        \item $a_n = \frac{n}{n+1}$
        \item $a_n = a_{n-1} + 1$ con $a_0 = 0$. 
    \end{itemize}
\end{frame}
%%%%%%%%%%%%%%%%%%%%%%%%%%%%%%%%%%%%%%%%%%%%%%%%%%%%%%%
\begin{frame}{Límite de sucesión}
    \begin{block}{}
        Una sucesión $a_n$ converge a $L$ si se cumple que
            $$ \forall \epsilon > 0, \exists n_0\in N, \forall n\geq n_0, |a_n - L | < \epsilon, $$
            y lo denotamos $\lim_{n\to\infty} a_n = L$. Al intervalo $(L-\epsilon, L+\epsilon)$ le llamamos una \emph{vecindad} de $L$.
    \end{block}
    \idea{Demostrar por definición que $a_n = 1/n$ converge a 0}
\end{frame}
%%%%%%%%%%%%%%%%%%%%%%%%%%%%%%%%%%%%%%%%%%%%%%%%%%%%%%%
\begin{frame}{No-existencia del límite}
    Podemos demostrar que un límite no existe usando la negación. Decimos que una sucesión \emph{no tiene límite} si se tiene que
            $$ \neg(\forall \epsilon > 0, \exists n_0\in N, \forall n\geq n_0, |a_n - L | < \epsilon) $$
            $$ =\exists \epsilon > 0, \forall n_0\in N, \exists n\geq n_0, |a_n - L | \geq \epsilon, $$
            \idea{Demostrar que $a_n=(-1)^n$ no tiene límite}
\end{frame}
%%%%%%%%%%%%%%%%%%%%%%%%%%%%%%%%%%%%%%%%%%%%%%%%%%%%%%%
%%%%%%%%%%%%%%%%%%%%%%%%%%%%%%%%%%%%%%%%%%%%%%%%%%%%%%%
\begin{frame}{Sucesiones monótonas}
    \begin{small} 
        \begin{block}{Sucesión monótona}
            Una sucesión $(a_n)_n$ es monótona (creciente) si se tiene que 
                $$ a_{n+1} \geq a_n. $$
                Decreciente se define análogamente.
        \end{block}
        \begin{block}{Teorema de sucesiones monótonas}
            Se $(a_n)_n$ es una sucesión monótona creciente y acotada superiormente, entonces es convergente y 
                $$ \lim_{n \to \infty}a_n = \sup_n a_n $$
        \end{block}
    *Se tiene además el análogo para decreciente y acotada inferiormente.
    \end{small}
\end{frame}
%%%%%%%%%%%%%%%%%%%%%%%%%%%%%%%%%%%%%%%%%%%%%%%%%%%%%%%
\begin{frame}[t]{Demostración}
    \begin{small}
        \begin{block}{Teorema de sucesiones monótonas}
            Se $(a_n)_n$ es una sucesión monótona creciente y acotada superiormente, entonces es convergente y $ \lim_{n \to \infty}a_n = \sup_n a_n $
        \end{block}
        \begin{enumerate}
            \item<1-> La sucesión está acotada por su supremo $s$.

                \only<1>{
                    Como $(a_n)_n$ es acotada, por Axioma del Supremo existe $s = \sup_n\{ a_n: n \in \N \}$. 
                }
                \item<2-> $s$ es nuestro candidato a límite. Buscamos $n_0$ tal que $a_n$ con $n\geq n_0$ estén cerca.
                    
                    \only<2>{
                        Como $s$ es supremo, $s-\epsilon$ \emph{no} es cota superior. Con esto, existe $n_0$ tal que 
                            $$ a_{n_0} > s - \epsilon. $$
                        La sucesión es creciente, luego 
                        $$ s-\epsilon < a_{n_0} < a_n \leq s < s-\epsilon. $$
                    }
                \item<3-> Vimos que $\forall \epsilon>0$, hay un $n_0$ a partir del cual $a_n$ está en una vecindad de $s$ ($|a_n -s |<\epsilon$). QED.
        \end{enumerate}
        \only<4>{Aplicación: Existe $e = \lim_n (1 + 1/n)^n$.}
    \end{small}
\end{frame}
%%%%%%%%%%%%%%%%%%%%%%%%%%%%%%%%%%%%%%%%%%%%%%%%%%%%%%%
\begin{frame}{Límites de funciones}
    \begin{small}
        Decimos que una función converge a $L$ cuando $x$ va a $x_0$ cuando...\footnote{Ambas son equivalentes}
        \begin{block}{Epsilon-delta ($\epsilon-\delta$)}
            $$\forall \epsilon>0, \exists \delta >0, \forall x\in \text{Dom}(f), |x-x_0|<\delta \Rightarrow |f(x) - f(x_0)| < \epsilon $$
        \end{block}

        \begin{block}{Sucesiones}
            $$\forall (x_n)_n \subset \text{Dom}(f)\setminus \{x_0\}, \lim_n x_n = x_0 \Rightarrow \lim_n f(x_m) = L $$
        \end{block}
        \idea{Demostrar con ambas definiciones que $\lim_{x\to 2}x^2=4$}
    \end{small}
\end{frame}
%%%%%%%%%%%%%%%%%%%%%%%%%%%%%%%%%%%%%%%%%%%%%%%%%%%%%%%
\darkSlide{En general, con sucesiones es más fácil.}
%%%%%%%%%%%%%%%%%%%%%%%%%%%%%%%%%%%%%%%%%%%%%%%%%%%%%%%
\begin{frame}{Límite superior / inferior}
    Dada sucesión $(x_n)_n$:
    \begin{block}{Límite superior}
        $$\limsup_{n\to\infty}x_n = \lim_{n\to\infty}\left(\sup_{k\geq n}x_n\right) $$
    \end{block}

    \begin{block}{Límite inferior}
        $$\limsup_{n\to\infty}x_n = \lim_{n\to\infty}\left(\inf_{k\geq n}x_n\right)$$
    \end{block}
    \idea{Encuentre el liminf y el limsup de $a_n=(-1)^n$.}
\end{frame}
%%%%%%%%%%%%%%%%%%%%%%%%%%%%%%%%%%%%%%%%%%%%%%%%%%%%%%%
\begin{frame}[t]{Ejercicio}
    \begin{block}{Límite superior}
        $$\limsup_{n\to\infty}x_n = \lim_{n\to\infty}\left(\sup_{k\geq n}x_n\right) $$
    \end{block}
    \begin{block}{Teorema de sucesiones monótonas}
        Se $(a_n)_n$ es una sucesión monótona creciente y acotada superiormente, entonces es convergente y $ \lim_{n \to \infty}a_n = \sup_n a_n $
    \end{block}
    Demuestre usando el teorema de sucesiones monótonas que el limsup siempre existe. Extienda su conclusión para el liminf. 
\end{frame}
%%%%%%%%%%%%%%%%%%%%%%%%%%%%%%%%%%%%%%%%%%%%%%%%%%%%%%%
\begin{frame}{Propiedades}
    \begin{enumerate}
        \item $\liminf_n a_n \leq \limsup_n a_n$
        \item Relación entre límites: 
            $$ \lim_n a_n = L \Leftrightarrow \liminf_n a_n = L = \limsup_n a_n $$
    \end{enumerate}
    \idea{Demostrar la primera}
\end{frame}
%%%%%%%%%%%%%%%%%%%%%%%%%%%%%%%%%%%%%%%%%%%%%%%%%%%%%%%
\begin{frame}
    \maketitle
\end{frame}
%%%%%%%%%%%%%%%%%%%%%%%%%%%%%%%%%%%%%%%%%%%%%%%%%%%%%%%
\end{document}
