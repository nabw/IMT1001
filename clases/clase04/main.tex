\documentclass[14pt,aspectratio=169,xcolor=dvipsnames]{beamer}
\usetheme{SimplePlus}
\usepackage{booktabs}
\usepackage{minted}

\title[short title]{Clase 04: El principio de inducción}
\subtitle{}
\author[NA Barnafi] {Nicolás Alejandro Barnafi Wittwer}
\institute[UC|CMM] 
{
    Pontificia Universidad Católica de Chile \\
    Centro de Modelamiento Matemático
}

\titlegraphic{
    \vspace{-1.8cm}
    \begin{flushright}
      \includegraphics[height=2.5cm]{../images/puc.png} 
    \end{flushright}
}

\date{16/10/2024}
%\setbeamercovered{transparent}

\begin{document}
%%%%%%%%%%%%%%%%%%%%%%%%%%%%%%%%%%%%%%%%%%%%%%%%%%%%%%%
\begin{frame}
    \maketitle
\end{frame}
%%%%%%%%%%%%%%%%%%%%%%%%%%%%%%%%%%%%%%%%%%%%%%%%%%%%%%%
\begin{frame}\frametitle{Motivación}
    \begin{itemize}
        \item Probar propiedades sobre números enteros
        \item Naturales: $1$ + \emph{sucesor}
        \item Extenderemos esta idea a demostraciones
    \end{itemize}
\end{frame}
%%%%%%%%%%%%%%%%%%%%%%%%%%%%%%%%%%%%%%%%%%%%%%%%%%%%%%%
\begin{frame}\frametitle{El principio}
    \begin{block}{}
        Suponer una afirmación $P(n)$ que depende de un número entero $n$. Esta afirmación es cierta para todo $n$ si las siguientes afirmaciones son ciertas:
        \begin{itemize}
            \item $P(1)$ es cierto 
            \item $P(k) \Rightarrow P(k+1)$ es cierto para todo $k$ entero
        \end{itemize}
    \end{block}
    \pause \idea{Demostrar que $n \leq 2^n$}
\end{frame}
%%%%%%%%%%%%%%%%%%%%%%%%%%%%%%%%%%%%%%%%%%%%%%%%%%%%%%%
\begin{frame}{Ejercicio}
    \begin{center}
        \idea{Demostrar que $n^2+n$ es par para todo $n$ entero}
\end{center}
\end{frame}
%%%%%%%%%%%%%%%%%%%%%%%%%%%%%%%%%%%%%%%%%%%%%%%%%%%%%%%
\begin{frame}{Variantes}
    \begin{itemize}
        \item Podemos cambiar $P(1)$ por  $P(n_0)$ (Demostrar)
        \item Definición por inducción
        \item Inducción fuerte
    \end{itemize}
\end{frame}
%%%%%%%%%%%%%%%%%%%%%%%%%%%%%%%%%%%%%%%%%%%%%%%%%%%%%%%
\begin{frame}{Definición por inducción}
    Considerar sucesión $a_1, a_2, \hdots$
    \begin{itemize}
        \item  $\sum_{i=1}^N a_i$
        \item $ \prod_{i=1}^N a_i$
        \item $x^n$
        \item $x!$
    \end{itemize}
    \pause
    \begin{center}
    \begin{minipage}{0.6\textwidth}
    \begin{block}{}
        \begin{itemize}
            \item Caso base: $\sum_{i=1}^1 = a_1$
            \item Paso inductivo: $\sum_{i=1}^N = \sum_{i=1}^{N-1} + a_n$
        \end{itemize}
    \end{block}
    \end{minipage}
    \end{center}
\end{frame}
%%%%%%%%%%%%%%%%%%%%%%%%%%%%%%%%%%%%%%%%%%%%%%%%%%%%%%%
\begin{frame}{Inducción fuerte}
    \begin{block}{}
        Suponer una afirmación $P(n)$ que depende de un número entero $n$. Esta afirmación es cierta para todo $n$ si las siguientes afirmaciones son ciertas:
        \begin{itemize}
            \item $P(1)$ es cierto 
            \item \{$P(n)$ vale para $n\leq k$\} $\Rightarrow P(k+1)$ es cierto para todo $k$ entero
        \end{itemize}
    \end{block}

\end{frame}
%%%%%%%%%%%%%%%%%%%%%%%%%%%%%%%%%%%%%%%%%%%%%%%%%%%%%%%
\begin{frame}
\end{frame}
%%%%%%%%%%%%%%%%%%%%%%%%%%%%%%%%%%%%%%%%%%%%%%%%%%%%%%%
\begin{frame}
\end{frame}
%%%%%%%%%%%%%%%%%%%%%%%%%%%%%%%%%%%%%%%%%%%%%%%%%%%%%%%
\begin{frame}
    \maketitle
\end{frame}
%%%%%%%%%%%%%%%%%%%%%%%%%%%%%%%%%%%%%%%%%%%%%%%%%%%%%%%
%%%%%%%%%%%%%%%%%%%%%%%%%%%%%%%%%%%%%%%%%%%%%%%%%%%%%%%
\end{document}
