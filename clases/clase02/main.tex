\documentclass[14pt,aspectratio=169,xcolor=dvipsnames]{beamer}
\usetheme{SimplePlus}
\usepackage{booktabs}
\usepackage{minted}

\title[short title]{Clase 03 Lógica Proposicional}
\subtitle{}
\author[NA Barnafi] {Nicolás Alejandro Barnafi Wittwer}
\institute[UC|CMM] 
{
    Pontificia Universidad Católica de Chile \\
    Centro de Modelamiento Matemático
}

\titlegraphic{
    \vspace{-1.8cm}
    \begin{flushright}
      \includegraphics[height=2.5cm]{../images/puc.png} 
    \end{flushright}
}

\date{16/10/2024}
%\setbeamercovered{transparent}

\begin{document}
%%%%%%%%%%%%%%%%%%%%%%%%%%%%%%%%%%%%%%%%%%%%%%%%%%%%%%%
\begin{frame}
    \maketitle
\end{frame}
%%%%%%%%%%%%%%%%%%%%%%%%%%%%%%%%%%%%%%%%%%%%%%%%%%%%%%%
\begin{frame}\frametitle{Clase de hoy}
    \begin{itemize}
        \item Proposiciones
        \item 'y', 'o', 'no', Tablas de verdad
        \item Implicancias
    \end{itemize}

    \vspace{1cm}
    \newref{Secciones I.1 y I.2, Eccles.}
\end{frame}
%%%%%%%%%%%%%%%%%%%%%%%%%%%%%%%%%%%%%%%%%%%%%%%%%%%%%%%
\begin{frame}{Objetos fundamentales de la matemática}
    \begin{itemize}
        \item Proposiciones $\to$ oraciones con valor de verdad (V$|$F)
            \begin{itemize}
                \item está lloviendo
                \item $\pi=3$
                \item $1+1=2$
                \item $n^2 -2n >0$ (depende de $n$)
            \end{itemize}
        \item Demostración
    \end{itemize}
\end{frame}
%%%%%%%%%%%%%%%%%%%%%%%%%%%%%%%%%%%%%%%%%%%%%%%%%%%%%%%
\begin{frame}{Negación ($\neg$)}
    $$ 1\neq 2 \Leftrightarrow  not( 1 = 2 )$$
    Tabla de verdad:
        \begin{center}
            \begin{tabular}{c | c}
                \toprule P & $\neg $ P  \\\midrule
                V & F \\
                F & V \\ \bottomrule
            \end{tabular}
        \end{center}
    Precedencia: potencia
\end{frame}
%%%%%%%%%%%%%%%%%%%%%%%%%%%%%%%%%%%%%%%%%%%%%%%%%%%%%%%
\begin{frame}{Disjunción ('o', $\vee$)}
    $$ ab = 0 \Leftrightarrow a=0 \text{ o } b=0 $$
    Tabla de verdad:
        \begin{center}
            \begin{tabular}{c c | c}
                \toprule P & Q & P $\vee$ Q \\\midrule
                V & V & V \\
                V & F & V \\
                F & V & V \\
                F & F & F  \\ \bottomrule
            \end{tabular}
        \end{center}

    Precedencia: $+$
\end{frame}
%%%%%%%%%%%%%%%%%%%%%%%%%%%%%%%%%%%%%%%%%%%%%%%%%%%%%%%
\begin{frame}{Conjunción ('y', $\wedge$)}
    $$ 0<1<2 \Leftrightarrow 0<1 \text{ y } 1<2 $$
    Tabla de verdad:
        \begin{center}
            \begin{tabular}{c c | c}
                \toprule P & Q & P $\wedge$ Q \\\midrule
                V & V & V \\
                V & F & F \\
                F & V & F \\
                F & F & F  \\ \bottomrule
            \end{tabular}
        \end{center}

    Precedencia: $*$
\end{frame}
%%%%%%%%%%%%%%%%%%%%%%%%%%%%%%%%%%%%%%%%%%%%%%%%%%%%%%%
\begin{frame}{Ejercicio}
    \begin{center}
        \idea{Tabla de verdad de: $ \neg P \vee Q$}
        \idea{Tabla de verdad de: $ P \vee \neq Q \wedge R $}
    \end{center}
\end{frame}
%%%%%%%%%%%%%%%%%%%%%%%%%%%%%%%%%%%%%%%%%%%%%%%%%%%%%%%
\begin{frame}{Ejemplos}
    \begin{itemize}
        \item $x \leq 2$
        \item $x \in (a,b)$
        \item $x \in [a,b]$
        \item $x\neq 1$
    \end{itemize}
\end{frame}
%%%%%%%%%%%%%%%%%%%%%%%%%%%%%%%%%%%%%%%%%%%%%%%%%%%%%%%
\begin{frame}{Implicancias}
    $$ n > 2 \Rightarrow n > 0 $$
    \begin{columns}
        \begin{column}{0.45\textwidth}
            Tabla de verdad:
            \begin{center}
                \begin{tabular}{c c | c}
                    \toprule P & Q & P $\Rightarrow$ Q \\\midrule
                    V & V & V \\
                    V & F & F \\
                    F & V & \alert{V} \\
                    F & F & V  \\ \bottomrule
                \end{tabular}
            \end{center}
        \end{column}
        \begin{column}{0.45\textwidth}
            \begin{itemize}
                \item Si P, luego Q
                \item P implica Q
                \item Q si P
                \item P solo si Q
                \item \alert{P es suficiente para Q}
                \item \alert{Q es necesaria para P}
            \end{itemize}
        \end{column}
    \end{columns}
\end{frame}
%%%%%%%%%%%%%%%%%%%%%%%%%%%%%%%%%%%%%%%%%%%%%%%%%%%%%%%
\begin{frame}{Ejercicios}
    \begin{itemize}
        \item $\neg \Rightarrow$
        \item $ P\Rightarrow Q$ equivale a $\neg P \vee Q$
        \item $ \Leftrightarrow $  (si y solo si, ssi, iff)
    \end{itemize}
\end{frame}
%%%%%%%%%%%%%%%%%%%%%%%%%%%%%%%%%%%%%%%%%%%%%%%%%%%%%%%
\begin{frame}\frametitle{Clase de hoy}
    \begin{itemize}
        \item Proposiciones
        \item 'y', 'o', 'no', Tablas de verdad
        \item Implicancias
    \end{itemize}

    \vspace{1cm}
    \newref{Secciones I.1 y I.2, Eccles.}
\end{frame}
%%%%%%%%%%%%%%%%%%%%%%%%%%%%%%%%%%%%%%%%%%%%%%%%%%%%%%%
\begin{frame}
    \maketitle
\end{frame}
%%%%%%%%%%%%%%%%%%%%%%%%%%%%%%%%%%%%%%%%%%%%%%%%%%%%%%%
%%%%%%%%%%%%%%%%%%%%%%%%%%%%%%%%%%%%%%%%%%%%%%%%%%%%%%%
\end{document}
