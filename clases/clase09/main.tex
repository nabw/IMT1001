\documentclass[14pt,aspectratio=169,xcolor=dvipsnames]{beamer}
\usetheme{SimplePlus}
\usepackage{booktabs}
\usepackage{minted}
\newcommand{\R}{\mathbb{R}}

\title[short title]{Clase 09: Acotamiento y axioma del supremo}
\subtitle{}
\author[NA Barnafi] {Nicolás Alejandro Barnafi Wittwer}
\institute[UC|CMM] 
{
    Pontificia Universidad Católica de Chile \\
    Centro de Modelamiento Matemático
}

\titlegraphic{
    \vspace{-1.8cm}
    \begin{flushright}
      \includegraphics[height=2.5cm]{../images/puc.png} 
    \end{flushright}
}

\date{07/04/2025}
%\setbeamercovered{transparent}

\begin{document}
%%%%%%%%%%%%%%%%%%%%%%%%%%%%%%%%%%%%%%%%%%%%%%%%%%%%%%%
\begin{frame}
    \maketitle
\end{frame}
%%%%%%%%%%%%%%%%%%%%%%%%%%%%%%%%%%%%%%%%%%%%%%%%%%%%%%%
\begin{frame}\frametitle{Temas de hoy}
    \begin{itemize}
        \item Acotamiento superior
        \item Acotamiento inferior
        \item Axioma del supremo (e ínfimo)
        \item Aplicaciones
    \end{itemize}

    \vspace{1cm}
    *Veremos todo en $\R$, pero son conceptos simples de extender

    \vspace{1cm}
    \newref{Spivak, Cálculo}
\end{frame}
%%%%%%%%%%%%%%%%%%%%%%%%%%%%%%%%%%%%%%%%%%%%%%%%%%%%%%%
\begin{frame}\frametitle{Acotamiento}
    \only<1>{
        \begin{block}{Cota superior}
            Decimos que $A\subset \R$ es superiormente acotado si se cumple que
                $$ \exists M\in \R, \forall x \in A, x\leq M. $$
                Llamamos a $M$ la \emph{cota superior} de $A$.
        \end{block}
        \idea{Encontrar cota superior de $(0,1)$ y de $(-\infty, 2)$}
    }
    \only<2>{
        \begin{block}{Cota inferior}
            Decimos que $A\subset \R$ es inferior acotado si se cumple que
                $$ \exists M\in \R, \forall x \in A, x\geq M. $$
                Llamamos a $M$ la \emph{cota inferior} de $A$.
        \end{block}
    }
    \only<3>{
        \begin{block}{Acotado}
            Si existe $M$ tal que $|x|\leq M$ para todo $x\in A\subset \R$, decimos que $A$ es \emph{acotado}.
        \end{block}
        \idea{Demostrar que acotado es equivalente a sup + inf acotado}
    }

\end{frame}
%%%%%%%%%%%%%%%%%%%%%%%%%%%%%%%%%%%%%%%%%%%%%%%%%%%%%%%
\begin{frame}{Máximos y mínimos}
    \begin{block}{Máximo}
        $A$ posee un máximo si posee una cota superior que pertenece a $A$
    \end{block}
    \begin{block}{Mínimo}
        $A$ posee un mínimo si posee una cota inferior que pertenece a $A$
    \end{block}
    \idea{Notar paralelo entre máximo/mínimo con intervalos abiertos/cerrados}
\end{frame}
%%%%%%%%%%%%%%%%%%%%%%%%%%%%%%%%%%%%%%%%%%%%%%%%%%%%%%%
\begin{frame}{Supremo e ínfimo}
    \only<1>{
        \begin{block}{Supremo}
            Dado $A\in\R$, $M$ es el supremo de $A$ si 
            \begin{itemize}
                \item $M$ es cota superior de $A$
                \item $s>M$ para cualquier otra cota superior $s$ de $A$
            \end{itemize}
            O sea, la menor cota superior de $A$.
        \end{block}
        \idea{El máximo de $(0,1)$ no existe, pero su supremo si.}
    }
    \only<2>{
        \begin{block}{Ínfimo}
            Dado $A\in\R$, $m$ es el ínfimo de $A$ si 
            \begin{itemize}
                \item $m$ es cota inferior de $A$
                \item $s<m$ para cualquier otra cota inferior $s$ de $A$
            \end{itemize}
            O sea, la mayor cota inferior de $A$.
        \end{block}
    }

\end{frame}
%%%%%%%%%%%%%%%%%%%%%%%%%%%%%%%%%%%%%%%%%%%%%%%%%%%%%%%
\begin{frame}{Propiedades}
    Dados dos conjuntos $A, B$ en $\R$, definimos
    $$ A+B = \{ x+y: x\in A, y \in B\} $$
    $$ A\cdot B = \{xy: x\in A, y \in B \} $$
    Luego:
    \begin{enumerate}
        \item $\sup(A+B) = \sup A + \sup B $
        \item $\sup(A\cdot B) = \sup A \sup B$ si $A, B\subset [0,\infty)$
    \end{enumerate}
    \idea{Demostrar la primera (probar ambas desigualdades)}
\end{frame}
%%%%%%%%%%%%%%%%%%%%%%%%%%%%%%%%%%%%%%%%%%%%%%%%%%%%%%%
\begin{frame}{Axioma del supremo}
    La existencia del supremo no se deduce de las propiedades de $\R$. 
    \begin{block}{Axioma del supremo}
        Todo conjunto acotado y no vacío posee un supremo
    \end{block}
    \pause Notar que eso implica el ínfimo ya que $\inf A = - \sup -A$.
\end{frame}
%%%%%%%%%%%%%%%%%%%%%%%%%%%%%%%%%%%%%%%%%%%%%%%%%%%%%%%
\begin{frame}{Aplicación: construcción de números}
    Qué es la raíz cuadrada de un real? 
    \pause en $\mathbb N$ y $\mathbb Q$ es claro. Demostración guiada:
    \begin{enumerate}
        \item Verificar que el conjunto $A=\{r\in \R: r^2\leq2\}$ tiene supremo, y definir $s = \sup A$. 
        \item Demuestre por contradicción que $s^2\geq 2$ considerando una constante positiva $\epsilon$ tal que $(s+\epsilon)^2 < 2$.
        \item Demuestre por contradicción que $s^2\leq 2$ análogamente con $2-\epsilon$
        \item Concluya que $s$ es el número buscado.
    \end{enumerate}
\end{frame}
%%%%%%%%%%%%%%%%%%%%%%%%%%%%%%%%%%%%%%%%%%%%%%%%%%%%%%%
\begin{frame}\frametitle{Temas de hoy}
    \begin{itemize}
        \item Acotamiento superior
        \item Acotamiento inferior
        \item Axioma del supremo (e ínfimo)
        \item Aplicación
    \end{itemize}
\end{frame}
%%%%%%%%%%%%%%%%%%%%%%%%%%%%%%%%%%%%%%%%%%%%%%%%%%%%%%%
\begin{frame}
    \maketitle
\end{frame}
%%%%%%%%%%%%%%%%%%%%%%%%%%%%%%%%%%%%%%%%%%%%%%%%%%%%%%%
%%%%%%%%%%%%%%%%%%%%%%%%%%%%%%%%%%%%%%%%%%%%%%%%%%%%%%%
\end{document}
