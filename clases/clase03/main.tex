\documentclass[14pt,aspectratio=169,xcolor=dvipsnames]{beamer}
\usetheme{SimplePlus}
\usepackage{booktabs}
\usepackage{minted}

\title[short title]{Clase 03 Demostraciones}
\subtitle{}
\author[NA Barnafi] {Nicolás Alejandro Barnafi Wittwer}
\institute[UC|CMM] 
{
    Pontificia Universidad Católica de Chile \\
    Centro de Modelamiento Matemático
}

\titlegraphic{
    \vspace{-1.8cm}
    \begin{flushright}
      \includegraphics[height=2.5cm]{../images/puc.png} 
    \end{flushright}
}

\date{16/10/2024}
%\setbeamercovered{transparent}

\begin{document}
%%%%%%%%%%%%%%%%%%%%%%%%%%%%%%%%%%%%%%%%%%%%%%%%%%%%%%%
\begin{frame}
    \maketitle
\end{frame}
%%%%%%%%%%%%%%%%%%%%%%%%%%%%%%%%%%%%%%%%%%%%%%%%%%%%%%%
\begin{frame}\frametitle{Clase de hoy}
    \begin{itemize}
        \item Demostración directa
        \item Demostración por contradicción
        \item Contrarrecíproca
    \end{itemize}

    \newref{Secciones I.3 y I.4, Eccles}
\end{frame}
%%%%%%%%%%%%%%%%%%%%%%%%%%%%%%%%%%%%%%%%%%%%%%%%%%%%%%%
\begin{frame}{Dem. directa}
    $$ P \Rightarrow Q$$
    \begin{columns}
        \begin{column}{0.45\textwidth}
            Tabla de verdad:
            \begin{center}
                \begin{tabular}{c c | c}
                    \toprule P & Q & P $\Rightarrow$ Q \\\midrule
                    V & V & V \\
                    V & F & F \\
                    F & V & \alert{V} \\
                    F & F & V  \\ \bottomrule
                \end{tabular}
            \end{center}
        \end{column}
        \begin{column}{0.45\textwidth}
            \begin{itemize}
                \item<2> Asumimos $P$ verdadero
                \item<2> Verificamos $Q$
            \end{itemize}
        \end{column}
    \end{columns}

\end{frame}
%%%%%%%%%%%%%%%%%%%%%%%%%%%%%%%%%%%%%%%%%%%%%%%%%%%%%%%
\begin{frame}{Ejercicios}
    \begin{center}
        \idea{Si $a,b > 0$, luego $a < b$ implica que $a^2 < b^2$}

        \idea{Si $a\in \mathbb R \setminus \{0\}$, luego $a^2 > 0$}
    \end{center}
\end{frame}
%%%%%%%%%%%%%%%%%%%%%%%%%%%%%%%%%%%%%%%%%%%%%%%%%%%%%%%
\begin{frame}{Demostrar $Q$ por contradicción}
    \begin{itemize}
        \item<1-> Suponer que $Q$ es falso
        \item<2-> Llegar a una contradicción
            \only<2>{
            \begin{flushright}
                \begin{tabular}{c | c}
                    \toprule P & P $\wedge$ $(\neg P)$ \\\midrule
                    V & F \\
                    F & F  \\ \bottomrule
                \end{tabular}
            \end{flushright}
            }
        \item<3-> Tabla de verdad (\alert{sup. Q falso}):
    \end{itemize}
        \only<3>{\begin{center}
            \begin{tabular}{c c | c}
                \toprule P & Q & $(\neg Q) \Rightarrow P$\\\midrule
                V & V & V \\
                V & F & \alert{V} \\
                F & V & V \\
                F & F & \alert{F}  \\ \bottomrule
            \end{tabular}
        \end{center}
        \hfill \alertGreen{Si $\neg P \Rightarrow Q $ es V, como $Q$ es F, luego $P$ es V.}
        }
\end{frame}
%%%%%%%%%%%%%%%%%%%%%%%%%%%%%%%%%%%%%%%%%%%%%%%%%%%%%%%
\begin{frame}{Ejemplo}

    \begin{center}
        \idea{$\sqrt 2$ es irracional}
    \end{center}

    \pause A qué números podemos extender esta demostración? 
\end{frame}
%%%%%%%%%%%%%%%%%%%%%%%%%%%%%%%%%%%%%%%%%%%%%%%%%%%%%%%
\begin{frame}{Contraproposición}
    Verifiquemos que $P\Rightarrow Q \Leftrightarrow (\neg Q)\Rightarrow (\neg P)$ \pause
    \begin{center}
        \idea{Aplicación: Si $a>b$, luego $ac \leq bc \Rightarrow c\leq 0$}
    \end{center}
\end{frame}
%%%%%%%%%%%%%%%%%%%%%%%%%%%%%%%%%%%%%%%%%%%%%%%%%%%%%%%
\begin{frame}
    \maketitle
\end{frame}
%%%%%%%%%%%%%%%%%%%%%%%%%%%%%%%%%%%%%%%%%%%%%%%%%%%%%%%
%%%%%%%%%%%%%%%%%%%%%%%%%%%%%%%%%%%%%%%%%%%%%%%%%%%%%%%
\end{document}
