\documentclass[14pt,aspectratio=169,xcolor=dvipsnames]{beamer}
\usetheme{SimplePlus}
\usepackage{booktabs}

\newcommand{\R}{\mathbb{R}}

\title[short title]{Clase 07 Funciones}
\subtitle{}
\author[NA Barnafi] {Nicolás Alejandro Barnafi Wittwer}
\institute[UC|CMM] 
{
    Pontificia Universidad Católica de Chile \\
    Centro de Modelamiento Matemático
}

\titlegraphic{
    \vspace{-1.8cm}
    \begin{flushright}
      \includegraphics[height=2.5cm]{../images/puc.png} 
    \end{flushright}
}

\date{26/03/2024}
%\setbeamercovered{transparent}

\begin{document}
%%%%%%%%%%%%%%%%%%%%%%%%%%%%%%%%%%%%%%%%%%%%%%%%%%%%%%%
\begin{frame}
    \maketitle
\end{frame}
%%%%%%%%%%%%%%%%%%%%%%%%%%%%%%%%%%%%%%%%%%%%%%%%%%%%%%%
\begin{frame}{Motivación}
    \begin{itemize}
        \item La función es uno de los objetos más fundamentals de la matemática
        \item Las empezamos a usar desde el día 0 
        \item .... sin una definición rigurosa
    \end{itemize}
    \vspace{1cm}
    \newref{Eccles, Cap. 8}
\end{frame}
%%%%%%%%%%%%%%%%%%%%%%%%%%%%%%%%%%%%%%%%%%%%%%%%%%%%%%%
%%%%%%%%%%%%%%%%%%%%%%%%%%%%%%%%%%%%%%%%%%%%%%%%%%%%%%%
\begin{frame}{Función}
    \begin{block}{}
        Dados conjuntos $X, Y$, una \emph{función} de $X$ a $Y$ es la asignación de un único elemento en $Y$ para cada $X$.
    \end{block}
    \begin{itemize}
        \item Al valor asignado a $x$ le llamamos $f(x)$ (\emph{valor de f en x})
        \item Se escribe también como aplicación: $x\mapsto f(x)$
        \item 
            $$ f: \underbrace{X}_\text{dominio} \to \underbrace{Y}_\text{codominio} $$
    \end{itemize}
\end{frame}
%%%%%%%%%%%%%%%%%%%%%%%%%%%%%%%%%%%%%%%%%%%%%%%%%%%%%%%
\begin{frame}{Ejemplos}
    \begin{itemize}
        \item $X=\{1,2\}, Y=\{2,3\}$, $f:X\to Y$ con $f(1)=2$, $f(2)=2$
        \item $f:\mathbb R \to \mathbb R$, $f(x) = x^2$
        \item $f:\mathbb R_+ \to \mathbb R$, $f(x) = x^2$
        \item $f:\mathbb R_- \to \mathbb R$, $f(x) = x^2$
        \item $f:(-1,2) \to \mathbb R$, $f(x) = x^2$
    \end{itemize}
\end{frame}
%%%%%%%%%%%%%%%%%%%%%%%%%%%%%%%%%%%%%%%%%%%%%%%%%%%%%%%
\begin{frame}{Funciones bien definidas}
    Una función debe tener una definición coherente para todos los puntos en su dominio: Sea $f:[0,\infty) \to \mathbb R$ dada por
    $$ f(x) = 
    \begin{cases}
        1/x & x\in (0,\infty) \\
        12  & x = 0
    \end{cases}
    $$
\end{frame}
%%%%%%%%%%%%%%%%%%%%%%%%%%%%%%%%%%%%%%%%%%%%%%%%%%%%%%%
\begin{frame}{Notaciones}
    \begin{small}
    \begin{itemize}
            \item Dos fnes son iguales ($f=g$) si toman el mismo valor en todo punto 
                \begin{flushright}
                    $f(x) = x^2-1, g(x) = (x+1)(x-1)$
                \end{flushright}
            \item Sean $f:X\to Y$, $A\subset X$. La \emph{restricción} de $f$ a $A$ se denota $f|_A$ y está dada por $f|_A:A\to Y$ con valores $f|_A(x)=f(x)$ para todo $x \in A$. 
            \item Si $A\subset B$, $x\in B\mapsto x\in A$ se llama \emph{restricción}
            \item Sean $f:X\to Y$, $g:Y\to Z$. Llamamos $g\circ f:X\to Z$ a la \emph{composición} de las funciones $f$ y $g$, dada por $g\circ f(x) = g(f(x))$. 
                \begin{flushright}
                    $f:(0,1)\to (1,2)$, $f(x) = x+1$, $g:(1,2)\to (1,4)$, $g(x) = x^2$
                \end{flushright}
            \item Si $A\subset B$, $x\in A\mapsto x\in B$ se llama \emph{inclusión}
            \item $I_X:X \to X$, $I_X(x) = x$ se llama \emph{identidad}
        \end{itemize}
    \end{small}
\end{frame}
%%%%%%%%%%%%%%%%%%%%%%%%%%%%%%%%%%%%%%%%%%%%%%%%%%%%%%%
\begin{frame}{Imagen de una función}
    \begin{itemize}
        \item Una función no tiene por qué llegar a todo su codominio:
            $$f:(0,1)\to \mathbb R, f(x) = x^2, f(x) = 2, \text{etc}$$
        \item El conjunto al que SI llega se llama imagen:
            $$ Im(f) = \{ f(x) | f \in X \} $$
        \item El grafo es "el dibujo" en el plano
            $$ G(f) = \{(x,y) \in X\times Y: y = f(x)\} = \{(x, f(x)): x\in X \}$$
    \end{itemize}
\end{frame}
%%%%%%%%%%%%%%%%%%%%%%%%%%%%%%%%%%%%%%%%%%%%%%%%%%%%%%%
\begin{frame}{Ejercicios}
    Dadas $f:X\to Y$, $g:Y\to Z$, $h:Z\to W$:
    \begin{itemize}
        \item $(h\circ g)\circ f = h\circ (g\circ f):X \to W$
        \item $f\circ I_X = f = I_Y\circ f: X\to Y$
    \end{itemize}
\end{frame}
%%%%%%%%%%%%%%%%%%%%%%%%%%%%%%%%%%%%%%%%%%%%%%%%%%%%%%%
\begin{frame}
    \maketitle
\end{frame}
%%%%%%%%%%%%%%%%%%%%%%%%%%%%%%%%%%%%%%%%%%%%%%%%%%%%%%%
%%%%%%%%%%%%%%%%%%%%%%%%%%%%%%%%%%%%%%%%%%%%%%%%%%%%%%%
\end{document}
