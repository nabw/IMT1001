\documentclass[14pt,aspectratio=169,xcolor=dvipsnames]{beamer}
\usetheme{SimplePlus}
\usepackage{booktabs}

\newcommand{\R}{\mathbb{R}}

\title[short title]{Clase 06 Cuantificadores}
\subtitle{}
\author[NA Barnafi] {Nicolás Alejandro Barnafi Wittwer}
\institute[UC|CMM] 
{
    Pontificia Universidad Católica de Chile \\
    Centro de Modelamiento Matemático
}

\titlegraphic{
    \vspace{-1.8cm}
    \begin{flushright}
      \includegraphics[height=2.5cm]{../images/puc.png} 
    \end{flushright}
}

\date{26/03/2024}
%\setbeamercovered{transparent}

\begin{document}
%%%%%%%%%%%%%%%%%%%%%%%%%%%%%%%%%%%%%%%%%%%%%%%%%%%%%%%
\begin{frame}
    \maketitle
\end{frame}
%%%%%%%%%%%%%%%%%%%%%%%%%%%%%%%%%%%%%%%%%%%%%%%%%%%%%%%
\begin{frame}{Motivación}
    \begin{itemize}
        \item Afirmaciones existenciales 
            \begin{flushright}
                \emph{Existe un elemento en el conjunto tal que ...}
            \end{flushright}
        \item Afirmaciones universales
            \begin{flushright}
                \emph{Para todo elemento en el conjunto es cierto que ...}
            \end{flushright}
    \end{itemize}
    \vspace{1cm}
    \newref{Eccles, Cap. 7}
\end{frame}
%%%%%%%%%%%%%%%%%%%%%%%%%%%%%%%%%%%%%%%%%%%%%%%%%%%%%%%
\begin{frame}{Cuantificador universal}
    $$ \forall a \in A, P(a) \Leftrightarrow \{a \in A|P(a)\} $$
    \emph{Para todo elemento $a$ en $A$, se tiene que $P(a)$}
    \pause
    \begin{itemize}
        \item $\forall a \in \mathbb R - \{0\}, a^2>0$
        \item $\{a\in \mathbb R | a^2 > 0 \} = \mathbb R - \{0\}$
        \item $a\in \mathbb R - \{0\} \Rightarrow a^2>0$
        \item $a$ es una variable \emph{muda}
    \end{itemize}
    \alert{Para demostrar estas propiedades se usa $a\in A\Rightarrow P(a)$}
\end{frame}
%%%%%%%%%%%%%%%%%%%%%%%%%%%%%%%%%%%%%%%%%%%%%%%%%%%%%%%
\begin{frame}{Cuantificador existencial}
    $$ \exists a \in A, P(a) \Leftrightarrow \{ a \in A | P(a) \} \neq \emptyset $$
    \emph{Existe un $a$ en $A$ tal que $P(a)$}
    \pause
    \begin{itemize}
        \item $\exists x \in \R, x^2-x=0$
        \item $\{x\in \R: x^2 = x\} \neq \emptyset$
    \end{itemize}
    \alert{Para demostrar estas propiedades, se exhibe un ejemplo}
\end{frame}
%%%%%%%%%%%%%%%%%%%%%%%%%%%%%%%%%%%%%%%%%%%%%%%%%%%%%%%
\begin{frame}{Notaciones}

    \begin{itemize}
        \item $\forall a \in A, \forall b \in A, P(a,b) \Leftrightarrow  \forall a,b \in A, P(a,b)$
        \item $\exists a \in A, \exists b \in A, P(a,b) \Leftrightarrow  \exists a,b \in A, P(a,b)$
            \idea{$\forall a,b \in \R, a<b \Rightarrow a^2 < b^2 $}
    \end{itemize}
\end{frame}
%%%%%%%%%%%%%%%%%%%%%%%%%%%%%%%%%%%%%%%%%%%%%%%%%%%%%%%
\begin{frame}{Negaciones}
    La negación de un cuantificador universal es el complemento del otro cuantificador universal
    \begin{itemize}
        \item $\neg(\forall x \in X, P(x)) = \exists x \in X, \neg P(x)$
        \item $\neg(\exists x \in X, P(x)) = \forall x \in X, \neg P(x)$

    \end{itemize}
    \pause
            \begin{flushright}
                \idea{Demostrar $\neg (\forall n\in \mathbb N, \exists p\in \mathbb N, n = 2p$)}
            \end{flushright}

\end{frame}
%%%%%%%%%%%%%%%%%%%%%%%%%%%%%%%%%%%%%%%%%%%%%%%%%%%%%%%
\begin{frame}
    \maketitle
\end{frame}
%%%%%%%%%%%%%%%%%%%%%%%%%%%%%%%%%%%%%%%%%%%%%%%%%%%%%%%
%%%%%%%%%%%%%%%%%%%%%%%%%%%%%%%%%%%%%%%%%%%%%%%%%%%%%%%
\end{document}
