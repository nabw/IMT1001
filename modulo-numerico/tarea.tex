\documentclass{article}
\usepackage{fullpage}
\usepackage{todonotes}
\usepackage{amsmath,amsfonts}
\newcommand{\R}{\mathbb{R}}
\newcommand{\code}[1]{\texttt{#1}}
\newcommand{\pts}[1]{[{\bf #1 puntos}] }
\newcommand{\note}[1]{\todo[inline,color=gray!20!white]{#1}}

\title{Tarea: Módulo de Análisis Numérico y Cómputo Científico}
\author{Nicolás Barnafi (\texttt{nicolas.barnafi@uc.cl})}
\begin{document}
\maketitle
\note{Esta tarea es individual y se debe entregar en el buzón habilitado para eso en Canvas con plazo máximo de entrega el miércoles 04/06/2025 al inicio de la ayudantía (i.e. 2 semanas de plazo). Tareas entregadas en \LaTeX (que sean comprensibles) tendrán una bonificación de 5 décimas. 

Para el desarrollo de los problemas, escriba el desarrollo en el mayor detalle posible, y defina todos los objetos matemáticos que use.

La tarea tiene 13 puntos, pero el puntaje máximo será de 9 puntos. Si responde más, su nota será dada por las mejores 9 respuestas. En cada pregunta, el puntaje dado será: 1 punto si está bien, 0.5 puntos si tiene un desarrollo concluyente pero tiene errores, y 0.0 puntos si tiene poco o nada de desarrollo.}



\begin{enumerate}
    \item En esta pregunta, veremos cómo aproximar numéricamente la función seno usando una serie de Taylor con respecto a $\pi$, ya que ahí conocemos los valores del seno.
        \begin{enumerate}
            \item\pts{1} Primero busquemos una forma de aproximar $\pi$, ya que es un número irracional y por lo tanto no se puede representar de manera exacta en un computador. Para ello, busque y muestre la serie de Ramanujan. Considerando la serie truncada hasta $N$ términos, muestre el resultado obtenido y compárelo con el valor de $\pi$ en \texttt{numpy.pi}. Cuántos términos de la serie considera prudente usar para aproximar a $\pi$? 
            \item\pts{1} Escriba la serie de Taylor del seno con respecto a $\pi$, y usando series truncadas encuentre el número $N$ de términos que necesita para que el error sea menor a un número dado $\epsilon$. Puede hacerlo numéricamente (o sea, programar una función que dado $\epsilon$ encuentre $k$)  o acotando el término del error. 
            \item\pts{1} Considerando que $\sin:\mathbb R\to\mathbb R$ es una función periódica, escriba una función en Python que, usando las series truncadas de Taylor (con respecto a $\pi$) y de Ramanujan, calcule el valor de $\sin(x)$ para cualquier $x$ en $\mathbb R$. 
        \end{enumerate}

    \item En el módulo numérico vimos que, usando diferencias finitas, podemos aproximar un operador diferencial con una matriz. En este ejercicio, haremos el ejercicio "al contrario", vale decir que intentaremos aproximar una integral. Considere $I=[a,b]$ y el operador integral $T:C(I) \to C^1(I)$ que toma una función continua y entrega una función diferenciable de la siguiente manera:
                    $$ Tf(x) = \int_a^x f(s)\,ds. $$

        \begin{enumerate}
            \item\pts{1} Considere una discretización del intervalo $I$ en $N$ intervalos $I^k$ con $I^k = (x^k, x^{k+1})$ y $|x^{k+1} - x^k| = \frac{b-a}{N}$, tal que la integral se puede aproximar con trapecios con
                $$ \int_a^b f(s)\,ds \approx T^Nf(b) := \sum_{k=0}^N \frac h 2 \left( f(x^k) + f(x^{k+1})  \right). $$
                Notar que esta es una definición para el operador $T^N$.  Considere la función exponencial y muestre que esta estrategia dada por la serie $T^Nf(b)$ converge al valor de la integral $Tf(b)$. Considere $a=0$, $b=1$.
            \item\pts{1} Adapte la aproximación de la pregunta anterior para escribir una matriz $\mathbf{T}$ tal que 
                $$ Tf(x^k) \approx [\mathbf{T} \vec f]_i, $$
                donde $\vec f$ es el vector de malla $\vec f = (f(x^0), \dots, f(x^N))$, y $[\mathbf T \vec f]_i$ es la fila $i$ del vector $\mathbf T \vec f$. 
            \item\pts{1} El Teorema Fundamental del Cálculo establece que si $f$ es diferenciable, se tiene que 
                    $$ f(x) = f(a) + \int_a^x f'(s)\,ds $$
                    Considere por simplicidad una función $f$ tal que $f(a) = 0$, y que $(a,b)=(0,1)$. Calcule la matriz de diferencias finitas $\mathbf{D}$ y verifique si en el discreto se cumple que para $f(a)$ se tiene
                    $$ \vec f \approx \mathbf{T} \mathbf{D} \vec f. $$
            \item\pts{1} Cómo modificaría su desarrollo para considerar que $f(a)\neq 0$? 
        \end{enumerate}

    \item Considere la siguiente ecuación diferencial: 
        $$
            \left\lbrace\begin{aligned}
                - \mu u'' + b u' + c u &= 1 && \text{en $\Omega$}\\
                u(L) = u(R) &= 0,
            \end{aligned}\right.
        $$
        donde $\mu, b,c\in \mathbb R$ y el dominio $\Omega$ está dado por $\Omega = (L, R)$, para $ L < R$. 
        \begin{enumerate}
            \item\pts{1} Use un cambio de variables para reescribir el problema en el dominio $\Omega = (0,1)$.
            \item\pts{1} Desde ahora en adelante, considere $\mu=0.1, b=c=1$, $L=-1$, $R=1$. Mostrar la matriz obtenida al discretizar la ecuación diferencial con diferencias finitas, usando un espaciado constante entre puntos de tamaño $h$, i.e. $x_{i+1} - x_i = h$.
            \item\pts{1} Considere $N$ el número de intervalos de su discretización. Con eso, grafique (en un mismo gráfico) la solución discreta del sistema lineal que aproxima la ecuación diferencial para $N\in\{4, 20, 1000\}$, e identifique claramente cada función. 
            \item\pts{1}  Considere como solución inicial un vector con 0 en todas sus componentes, y grafique el error en cada iteración del método de (i) Richardson, (ii) Jacobi y (iii) Gauss-Seidel. Use como criterio de parada la norma del error relativo, con una tolerancia de $10^{-5}$. Para calcular el error relativo en cada iteración, debe considerar el residuo $r^k$ de la iteración y el residuo del vector inicial $r^0$. Luego, el error relativo está dado por 
                    $$ \frac{|r^k|}{|r^0|}. $$
            \item\pts{1} Muestre cómo varía el número de iteraciones requeridas para converger al variar el parámetro $\mu$ en $\{100,10,1,0.1,0.01,0.001,0.0001\}$. Puede elegir cualquiera de los métodos convergentes (si es que es más de uno).
            \item\pts{1} Justifique teóricamente por qué algunos métodos convergen y otros. Para ello, calcule la norma de la matriz de transferencia $\|B_\omega\|$.
    \end{enumerate}
\end{enumerate}

\end{document}
